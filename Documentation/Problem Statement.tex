\documentclass[12pt]{article}

\usepackage{xcolor} % for different colour comments

%% Comments
\newif\ifcomments\commentsfalse

\ifcomments
\newcommand{\authornote}[3]{\textcolor{#1}{[#3 ---#2]}}
\newcommand{\todo}[1]{\textcolor{red}{[TODO: #1]}}
\else
\newcommand{\authornote}[3]{}
\newcommand{\todo}[1]{}
\fi

\newcommand{\wss}[1]{\authornote{magenta}{SS}{#1}}
\newcommand{\ag}[1]{\authornote{blue}{AG}{#1}} % Shafeeq
\newcommand{\kp}[1]{\authornote{blue}{KP}{#1}} % Keyur
\newcommand{\sr}[1]{\authornote{blue}{SR}{#1}} % Alex

\begin{document}

\title{Problem Statement for Nibbles} 
\author{Alex Guerrero - guerreap\\Keyur Patel - patelk36\\Shafeeq Rabbani - rabbass}
\date{\today}
	
\maketitle
	Games are designed for multiple reasons but the main overarching purpose is for end users to have fun. Whether it be taking a break from doing work or boredom, video games offer an outlet to another realm where you can compete and challenge yourself, and potentially others as well.\par

Using open source code, we will recreate the classic snake game making the code more modular and thus maintainable for future use and modification. The program will be designed for the purpose of running on personal computers and laptops.\par

Stakeholders for this project include team members, professor, teaching assistant and fellow classmates.\par 

Team members are responsible redesigning and redeveloping the snake game and ensuring all of its documentation is present for future developers to make modifications. The team will be supervised by the professor and teaching assistant who will be providing feedback and evaluations for each deliverable. Classmates will have access to the snake game so that they may enjoy this classic arcade game.\par


\noindent \wss{This is an example comment.  You can turn comments off by replacing
  commentstrue by commentsfalse.}\\
\ag{Sample comment}\\
\kp{Sample comment}\\
\sr{Sample comment}

\end{document}










